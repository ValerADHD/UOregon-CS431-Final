\title{Rendering and Optimization of Gaussian Radiance Fields}

\author{Valerie Dagley, Cameron Durbin, Andrew Rehmann}

\newcommand{\abstractText}{\noindent
Abstract goes here.
}

%%%%%%%%%%%%%%%%%
% Configuration %
%%%%%%%%%%%%%%%%%

\documentclass[12pt, a4paper, twocolumn]{article}
\usepackage{xurl}
\usepackage[square, comma, numbers, sort]{natbib}
\setcitestyle{numbers, open={[}, close={]}}
\usepackage{abstract}
\renewcommand{\abstractnamefont}{\normalfont\bfseries}
\renewcommand{\abstracttextfont}{\normalfont\small\itshape}
\usepackage{lipsum}

\usepackage{titlesec}

\titleformat{\section}
  {\normalfont\fontsize{14}{14}\bfseries}{\thesection}{1em}{}


%%%%%%%%%%%%%%
% References %
%%%%%%%%%%%%%%

% If changing the name of the bib file, change \bibliography{test} at the bottom
\begin{filecontents}{workscited.bib}
@misc{kerbl20233d,
  title={3D Gaussian Splatting for Real-Time Radiance Field Rendering}, 
  author={Bernhard Kerbl and Georgios Kopanas and Thomas Leimkühler and George Drettakis},
  year={2023},
  eprint={2308.04079},
  archivePrefix={arXiv},
  primaryClass={cs.GR}
}
@misc{barron2022mipnerf,
  title={Mip-NeRF 360: Unbounded Anti-Aliased Neural Radiance Fields}, 
  author={Jonathan T. Barron and Ben Mildenhall and Dor Verbin and Pratul P. Srinivasan and Peter Hedman},
  year={2022},
  eprint={2111.12077},
  archivePrefix={arXiv},
  primaryClass={cs.CV}
}
\end{filecontents}

% Any configuration that should be done before the end of the preamble:
\usepackage{hyperref}
\hypersetup{colorlinks=true, urlcolor=blue, linkcolor=blue, citecolor=blue}

\begin{document}

%%%%%%%%%%%%
% Abstract %
%%%%%%%%%%%%

\twocolumn[
  \begin{@twocolumnfalse}
    \maketitle
    % \begin{abstract}
    %   \abstractText
    %   \newline
    %   \newline
    % \end{abstract}
  \end{@twocolumnfalse}
]

%%%%%%%%%%%
% Article %
%%%%%%%%%%%

\section{Project Area}
This is a fairly new subject when it comes to research.
Our main source of information comes from a recent paper,
3D Gaussian Splatting for Real-Time Radiance Field Rendering
\cite{kerbl20233d}. Other papers have detailed similar approaches to 
modeling of scenes, such as Neural Radiance Fields (NeRFs) \cite{barron2022mipnerf},
or DeepVoxels \cite{sitzmann2019deepvoxels}, but the novel
approach of this paper is its use of gaussians to encapsulate 
scene information and its fast rasterization.

The general approach they followed is to first instantiate 3D gaussians
based on sparse point clouds created with a process called structure from motion (SfM) \cite{Ko2016PointCG}.
Then, the gaussians are split and duplicated until they more densely fill the space,
later on being moved, removed, or split as needed. The gaussians are optimized
to fit the scene by iterations of rendering (rasterization) and gradient descent
against the known images of the scene. One of the main advances of this paper in particular
was their fast implementation of rasterization, which allows many more ``splats''
to receive gradients. This greatly speeds up training time as well as accuracy
of the final scene.

\section{Motivation}
(Valerie, maybe you can speak to this a bit more)
(Any ideas of exactly what sort of motivation we're talking about here?)

We chose this topic for our project at least partially because it appeals to the 
mixed strengths of our group, as well as to the desire to work on new and novel material.
Valerie's main focus of interest is graphics and she had already skimmed the paper on her own.
Cameron has prior experience when it comes to machine learning and gradient descent optimization (?).
Andrew is relatively inexperienced when it comes to either of these fields. However, as a math major,
he was interested in diving deeper into the world of gradient descent optimization, especially
as a way to manipulate 3D gaussians. 

\section{Directions of Investigation}
As discussed by the paper in section 8, one possible section of 
improvement would be to implement the optimization steps of the
code entirely in C. They note that 
\begin{quote}
  ``The majority (~80\%) of our training time is spent in Python code,
  since we built our solution in PyTorch to allow our method to be
  easily used by others. Only the rasterization routine is implemented
  as optimized CUDA kernels. We expect that porting the remaining
  optimization entirely to CUDA, as e.g., done in InstantNGP [Müller
  et al . 2022], could enable significant further speedup for applications
  where performance is essential.''
\end{quote}


\section{Expected Results}



%%%%%%%%%%%%%%
% References %
%%%%%%%%%%%%%%

\nocite{*}
\bibliographystyle{plainnat}
\bibliography{workscited.bib}

\end{document}
